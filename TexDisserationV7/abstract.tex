\begin{abstract}
An important application of wireless sensor networks is to observe the physical world. This dissertation focuses on observation methods by wireless sensor networks (WSNs), especially the unique issues due to the properties of WSNs.

    Mobility is an important feature of WSNs. Since the quality of the observation on the physical world depends on the positions of the sensors, the optimum sensor positions depend on the models of the physical systems. Mobile sensor networks can provide better estimates, since the sensor nodes can maneuver to the optimal sensing positions.
       In Chapter~\ref{s:MasnetIROS}, we study a problem called trajectory optimization for observation on Distributed Parameter Systems (DPSs). Given a model of a distributed parameter system, we proposed a numeric method to compute the optimal trajectory of mobile sensors, which are mounted on differentially-driven robots, such that the observation on the system parameter is optimized.

    Energy efficiency is the key challenge and the central research topic for WSNs. For WSN-based observation system, it is important to balance the tradeoff between the energy efficiency and estimation precision. Energy costs can be reduced significantly by selecting proper sensors. In Chapter~\ref{s:COSS}, we propose a convex optimal sensor selection (COSS) algorithm to choose proper sensors to observe parameters of physical systems, such as the position of a target. After introducing several approximations, we formulate the sensor selection problem as a convex optimization problem. Unlike common optimization formulations, we do not explicitly minimize the number of selected sensors. Our cost function is constructed to minimize estimation errors. However, due to the properties of the convex optimization, the number of selected sensors is minimized simultaneously during the time when the estimation error is minimized, which is referred as the implicit optimal sensor selection (IOSS). Our analysis provides some guidance on how to design and tune the IOSS methods. We prove that those IOSS methods not only select the minimum number of sensors that allowed by the Carath\'{e}odory's theorem, but also approximate the minimal estimation error predicted by the Cram\'{e}r-Rao lower bound (CRLB). In addition, we have tested the COSS algorithm by extensive simulation and hardware experiments. Those experiments are conducted under realistic conditions that are close to the real-world engineering practices.

    Localization is a fundamental problem for WSNs.
    The sensors' positions are required by most of the high-level WSN applications. In Chapter~\ref{s:loc}, an asynchronous TDOA localization method is proposed. When there are a large number of sensor nodes whose positions are unknown, asynchronous TDOA methods reduce communication costs as well as the beacon nodes' hardware costs. The positioning error of the localization system is affected by the placement of the beacon notes. In order to guarantee the robustness of the localization system, the beacon placement problem is formulated as a min-max optimization. The optimization is solved by semi-infinite programming method.

    Although the above problems are not identical, they share the same theoretical framework, i.e, optimal experimental design. They are formulated as convex optimization on cost functions based on the Fisher information matrix (FIM). FIM and convex analysis are promising mathematical tools for WSN-based signal processing.
    % The properties of convex optimization play important roles during the procedure to solve those practical problems.
\end{abstract}



