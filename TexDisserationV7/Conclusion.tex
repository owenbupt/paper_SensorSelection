\chapter{Conclusion}\label{s:dissConclusion}
Wireless sensor network (WSN) is an important technology for physical world observation. While WSN has significant advantages over the traditional wired data collection systems, it is facing unique challenges. For example, the massive and densely deployed WSN could provide more cost-efficient and fine-grid sensing on physical phenomena, comparing to the traditional methods. The mobilities of the wireless sensor nodes may further enhance the performance on the observation. To facilitate the above applications, many challenges must be addressed, including energy efficient observation, localization, scalability, fault tolerance, etc. Among them, energy efficiency is a central challenge that has caught much attentions and been attacked from different aspects, but is still under active research. Much of this dissertation is motivated by reducing energy costs.
    This discussion discusses on several key challenges of WSNs within a unified theoretical framework, i.e, the theory of optimal experimental design.

Our first challenge is to take the mobilities of sensors for better observation on physical phenomena. In order to drive mobile sensors along the optimum trajectories for parameter estimations on distribution parameter systems (DPSs), a numerical optimal control method is proposed.


    The second challenge is to design energy efficient observation methods. A sensor selection algorithm named COSS (Convex Optimal Sensor Selection) is designed and analyzed. We prove that there is a class of Implicit Optimal Sensor Selection (IOSS) methods, which includes the COSS algorithm and can achieve the optimal estimation predicted by the CRLB (Cram\'{e}r-Rao Lower Bound) with the minimal number of sensors that allowed by the Carath\'{e}odory's theorem. The robustness and efficiency of the COSS are tested extensively by simulation and hardware experiments.


        The last but not the least important challenge is WSN localization. Motivated by reducing hardware and energy costs, an asynchronous TDOA (Time Difference of Arrival) method is proposed, namely, phase of arrival method, which can be converted into the standard TDOA problem formulation.
      Based on the TDOA method, two fast beacon placement optimization algorithms are developed based on SIP (semi-infinite programming). The methods are applicable not only to the proposed asynchronous TDOA, but also generic TDOA and TOA (Time of Arrival) as well. Comparing to the related beacon placement methods, the proposed approaches are fast and have no constraints the shapes of the beacon deployment domains.

    All those challenges are formulated as convex optimizations on cost functions based on the FIM (Fisher information matrix), which plays a key role in the theory of optimal experimental design. The abstractions of the three problems in this dissertation share the same theoretical framework, and highly related to each other.
    % We conclude that FIM and convex analysis are promising mathematical tools for WSN-based signal processing.

